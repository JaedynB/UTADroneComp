% This file contains the text portion of the Background portion of the project charter

% 5 BACKGROUND
% A new paragraph with tab just needs an empty line
As of 2025, unmanned aircraft systems are expected to create more than 100,000 jobs, according to a study by the Association for Unmanned Vehicle Systems International (AUVSI) \cite{AUVSI}. Having said that, it is clear from recent war zones like Ukraine that drones have been used effectively. However, it is usually necessary for humans to control drones. As warfare becomes more dangerous and technologies advance, militaries are looking for ways to minimize their casualties. Rather than putting their troops in harm's way, military leaders want autonomous vehicles that can take down hostile targets. It is Raytheon's goal to support student research and innovation in autonomous vehicle design by sponsoring them in this showcase competition. As a result, Raytheon can scout and recruit top graduates from universities participating in the showcase. To simulate real-world target recognition for unmanned systems, students will identify ArUco markers placed on moving ground vehicles. Furthermore, students will gain industry insight while collaborating across multiple teams and coordinating for an eight month project that requires extensive documentation and coordination.

In addition to providing Raytheon with technology for national defense purposes, the AUVSI report estimates that the integration of these systems with the National Airspace System (NAS) between 2015 and 2025 will generate more than 80 billion dollars \cite{AUVSI}. This is due to the many applications for UAS in the following areas: wildfire mapping, agriculture, disaster management, law enforcement, cinema, environmental monitoring, and freight transport \cite{AUVSI}. There are no prior relationships between Raytheon and the student CSE team. Nevertheless, Raytheon may find students they wish to hire after graduation through its sponsorship of this project.
