% This file contains the text portion of the Background portion of the project charter


% 5 BACKGROUND
As of 2025, unmanned aircraft systems are expected to create more than 100,000 jobs, according to a study by the Association for Unmanned Vehicle Systems International (AUVSI) (credit here: Robert found this, I will add it later). As warfare becomes more dangerous and technologies advance, militaries are looking for ways to minimize their losses. Rather than putting their troops in harm's way, military leaders want autonomous vehicles that can take down hostile targets. It is Raytheon's goal to support student research and innovation in autonomous vehicle design by sponsoring them in this showcase competition. As a result, Raytheon is able to scout and recruit top graduates from universities participating in the showcase. Students will have to research and develop algorithms that use computer vision. In addition to developing algorithms that use computer vision, students will gain industry insight while collaborating across multiple teams and coordinating for a long project that requires extensive documentation and coordination.

% A new paragraph with tab just needs an empty line
In addition to providing Raytheon with technology for national defense purposes, the AUVSI report estimates that the integration of these systems with the National Airspace System (NAS) between 2015 and 2025 will generate more than 80 billion dollars. This is due to the many applications for UASs in the following areas: wildfire mapping, agriculture, disaster management, law enforcement, cinema, environmental monitoring, and freight transport (cite: Robert). There has been no prior relationship between the software development team and Raytheon, but this is expected to develop in the future.
