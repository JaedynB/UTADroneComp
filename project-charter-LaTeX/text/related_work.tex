% This file contains the text portion of the Related Work portion of the project charter

% 6 RELATED WORK
In order to complete this project, a drone that is autonomous will need to be built. A camera on the drone must identify ArUco markers, and a laser must target Ace Combat sensors placed on top of targeted ground vehicles. Moreover, the drone must be equipped with a GPS \& RTK system. UTA's CSE department has a long history of working on drone projects and has curated a list of parts that have proven competitive in competitions of this type.

A similar project was completed by the UTA CSE team for the 2021-2022 year for Raytheon, which included autonomous flight, GPS, and RTK systems. As a result of these similarities, their project will prove useful for building the 2023 drone and figuring out what parts would work best. Since their drone did not have the capability of targeting moving vehicles and sensors with lasers, it can serve as a basis for the 2023 drone, but additional sources to do the other tasks will need to be found \cite{Lotspeich}.

The first major milestone to completing this project is to build the drone. It is also necessary to learn how to connect the drone to Mission Planner to map out its flight patterns. The following is an example of how a drone is constructed and how it is connected to Mission Planner \cite{UoBDASAR}. They have a Raspberry Pi onboard their drone, but the goal is to eliminate the use of a single-board computer attached to the drone and instead connect to it wirelessly, which will reduce the drone's weight and eliminate the constraints that come with single-board computers. Another difference this year is the use of ArUco markers on top of moving ground vehicles to identify moving targets. It is possible to accomplish this task using a popular library called OpenCV, which has detailed documentation \cite{Bradsky}. Another source explains how to use OpenCV to detect ArUco markers and how to land on them in video \cite{Fiorenzani}. An additional source of information on the implementation aspect of OpenCV to identify ArUco markers is helpful for this project.

The University of Texas at Dallas (UTD) won the 2022 Raytheon drone competition. A description and diagram of the winning drone can be found on UTD's website. This drone contained two LiDAR sensors \cite{Posamentier}. Due to the radar spinning extremely fast to scan the surrounding area, LiDAR creates a lot of inertia. The drone being built by the UTA CSE team for the 2023 competition will avoid using rotational LiDAR. A LiDAR sensor will be used to measure the drone's height and speed. To eliminate the complications that can arise with omnidirectional LiDAR scanners, a unidirectional and non-rotational LiDAR sensor will be used. By minimizing inertial forces caused by omnidirectional LiDAR scanners, the team can control the drone more easily because it does not exert a lot of inertial forces.
