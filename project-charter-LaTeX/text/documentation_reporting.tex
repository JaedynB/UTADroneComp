% This file contains the text portion of the Documentation & Reporting portion of the project charter

\subsection{Major Documentation Deliverables}

\subsubsection{Project \& Charter}
The initial version of the Project Charter will be uploaded on October 2nd 2022. The final version of the Project charter will be uploaded on April 30th 2023. There is a possibility that the charter will be updated sprint to sprint as more information is received. Sections such as 14.2, 14.2.1, 14.2.5, etc, are most likely to be updated.

% This section needs to not be a subsection of the above subsection
\subsubsection{Systems \& Requirements \& Specification}	
The initial version of the Project Charter will be uploaded on October 2nd 2022. The final version of the Project charter will be uploaded on April 30th 2023. There is a possibility that the charter will be updated sprint to sprint as more information is received. Sections such as 14.2, 14.2.1, 14.2.5, etc, are most likely to be updated.

\subsubsection{Architectural Design Specification}

\subsubsection{Detailed Design Specification}

\subsection{Recurring Sprint Items}
This is the first sprint, so a projected recurring sprint item will be provided instead. Component testing would be the most likely recurring sprint item. As the team has more sprints moving forward an actual recurring sprint item(s) will be provided.

\subsubsection{Product Backlog}

\subsubsection{Sprint Planning}
There will be eight sprints throughout the duration of this project. Each of them would be planned taking into account lessons learned from previous sprints. The following methods will also be implemented:
\begin{itemize}
  \item Planning a sprint meeting after examining the team's availability
  \item Go over backlog and assigning ownership of tasks to team members
  \item Confirm new issues, impacts, and dependencies
  \item Reach a group consensus on time estimations
\end{itemize}

\subsubsection{Sprint Goal}
The sprint goal will be determined by the team during a meeting before the start of the sprint. The best way to keep the project stakeholders involved is by giving a report on our sprint goal to our sponsor on the first Friday of every sprint.

\subsubsection{Sprint Backlog}
{An excel spreadsheet will be used to maintain our sprint backlog, update time estimates, and display a burndown chart.}

\subsubsection{Task Breakdown}
Time spent on tasks will be logged by each individual within the team on our Excel spreadsheet.

\subsubsection{Sprint Burn Down Charts}
When each team member logs their hours worked for each task in the Excel spreadsheet, there will be a Burn Down chart that gets updated automatically. The sprint spreadsheet will include specific sprint backlogs in which the general tasks for that sprint are divided into smaller parts. This will indicate which team member(s) worked on each subtask. The subtasks are not recorded per hour, yet. The total number of hours expended by each team member is included in a table for every sprint backlog.


\subsubsection{Sprint Retrospective}
Following each sprint, the team will hold a retrospective meeting to discuss what went well and what could be improved. We will assign team tasks to each team member based on the sprint goals. Assigning tasks will keep team members accountable. Each task will be due at the end of each sprint.

\subsubsection{Individual Status Reports}
Team meetings will be held on Discord. The Discord server has a channel for the team members to post their notes and resources. Team meetings will be held on the meeting channel. The meeting and general channels are expected to serve as a space for posting each individual's progress and concerns.

\subsubsection{Engineering Notebooks}
Software development team members are expected to update their engineering notebooks at the end of each sprint in order to hold each other accountable for their work. It is essential that each team member is able to communicate effectively regarding their engineering notebooks in order to hold one another accountable. A mid-sprint meeting can be used to discuss the targeted number of pages for the current sprint. To ensure quality control, the SCRUM master must sign off as a witness for each engineering notebook page.

\subsection{Closeout Materials}

\subsubsection{System Prototype}

\subsubsection{Project Poster}
In addition to an architectural diagram, the poster will also explain Python use cases and describe the components of the drone.

\subsubsection{Web Page}

\subsubsection{Demo Video}
As certain parts of the project become demonstrable, demo videos will be provided. Demo videos will demonstrate key aspects of the project and how they behave and respond in real-world environments.

\subsubsection{Source Code}
Version control and source code maintenance will be handled by the team using Git and GitHub. The source code will not be accessible to the public at any time during the project's lifecycle.

\subsubsection{Source Code Documentation}
Sphinx, which will be able to generate all documentation from the source code, will be able to create all documentation using Python's code standard. Documentation would be produced in LaTeX.

\subsubsection{Hardware Schematics}

\subsubsection{User Manual}
This project's user manual will be available at a later date. A setup video may be included in the user manual explaining how the hardware is configured and how to run the code. In the case a video is included, the visual instructions will show how to run the project in a simple and straightforward manner.